\chapter*{Вывод}
\addcontentsline{toc}{chapter}{Вывод}

Функции обработчика прерывания от системного таймера в защищенном режиме для
семейства ОС \texttt{Windows} и для семейства OC \texttt{UNIX/Linux} очень
похожи по своим действиям. Они выполняют схожие задачи:

\begin{itemize}
    \item инициализируют (но не выполняют) отложенные действия, относящиеся к
          работе планировщика, такие как пересчет приоритетов;
    \item выполняют декремент счетчиков времени: часов, таймеров, будильников
          реального времени, счетчиков времени отложенных действий.
    \item выполняют декремент кванта (текущего процесса в \texttt{Linux},
          текущего потока в \texttt{Windows}).
\end{itemize}

Обе системы являются системами разделения времени с динамическими приоритетами
и вытеснением, пересчёт динамических приоритетов в данных системах можно
описать следующим образом:
\begin{itemize}
    \item В \texttt{UNIX/Linux} приоритет процесса характеризуется текущим
          приоритетом и приоритетом процесса в режиме задачи. Приоритет
          пользовательского процесса --- процесса в режиме задачи --- может
          быть динамически пересчитан в зависимости от фактора любезности и
          величины использования процессора, в то время как приоритеты ядра
          являются фиксированными величинами.
    \item При создании процесса в \texttt{Windows}, ему назначается приоритет,
          обычно называемый базовым. Приоритеты потоков определяются
          относительно приоритета процесса, в котором они создаются. Приоритет
          потока пользовательского процесса может быть пересчитан динамически.
\end{itemize}
