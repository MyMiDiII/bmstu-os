\chapter*{Вывод}
\addcontentsline{toc}{chapter}{Вывод}

Обработчик прерывания от системного таймера в защищенном режиме для ОС
семейства Windows и для OC семейства UNIX/Linux выполняют схожие задачи:
\begin{itemize}
    \item инициализируют отложенные действия, относящиеся к
          работе планировщика;
    \item выполняют декремент счетчиков времени: часов, таймеров, будильников
          реального времени, счетчиков времени отложенных действий.
    \item выполняют декремент кванта (текущего процесса в Linux,
          текущего потока в Windows).
\end{itemize}

Обе системы являются системами разделения времени с динамическими приоритетами
и вытеснением, пересчёт динамических приоритетов в данных системах можно
описать следующим образом:
\begin{itemize}
    \item При создании процесса в Windows, ему назначается приоритет,
          обычно называемый базовым. Приоритеты потоков определяются
          относительно приоритета процесса, в котором они создаются. Приоритет
          потока пользовательского процесса может быть пересчитан динамически.
    \item В UNIX/Linux приоритет процесса характеризуется текущим
          приоритетом и приоритетом процесса в режиме задачи. Приоритет
          пользовательского процесса --- процесса в режиме задачи --- может
          быть динамически пересчитан в зависимости от фактора любезности и
          величины использования процессора, в то время как приоритеты ядра
          являются фиксированными величинами.
\end{itemize}
