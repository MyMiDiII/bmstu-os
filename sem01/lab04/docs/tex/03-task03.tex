\section*{Задание №3}

Процессы-потомки переходят на выполнение других программ, которые передаются
системному вызову \textit{exec()}. Один процесс выполняет программу,
осуществляющую поиск расстояния Левенштейна между двумя строками, другой --
программу, осуществляющую поиск недостижимых вершин из данной в ориентированном
графе. Предок ждет завершения своих потомков с анализом кодов завершения. На
экран выводятся соответствующие сообщения.

\begin{lstinputlisting}[
    caption={Системный вызов exec()},
	label={lst:exec}
]{../../src/task03.c}
\end{lstinputlisting}

\img{170mm}{img03}{Демонстрация работы программы (задание №3)}{img03}
