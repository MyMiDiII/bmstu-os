\documentclass[a4paper,14pt]{extreport}

% table of contents
\usepackage{tocloft}
\renewcommand{\cfttoctitlefont}{\hfill\large\bfseries}
\renewcommand{\cftaftertoctitle}{\hfill\hfill}
\renewcommand{\cftchapleader}{\cftdotfill{\cftdotsep}}
\setlength\cftbeforetoctitleskip{-22pt}
\setlength\cftaftertoctitleskip{15pt}

% searchable and copyable
\usepackage{cmap}

% Times New Roman
\usepackage{pscyr}
\renewcommand{\rmdefault}{ftm}

% Links
\def\UrlBreaks{\do\/\do-\do\_}
\usepackage[nottoc]{tocbibind} % for bib link
\usepackage[numbers]{natbib}
\renewcommand*{\bibnumfmt}[1]{#1.}

%% Images
\usepackage{graphicx}
\usepackage{wrapfig}

\newcommand{\img}[4] {
	\begin{figure}[ht!]
		\center{
            \includegraphics[height=#1]{../data/img/#2}
            \caption{#3}
            \label{img:#4}
        }
	\end{figure}
}

%% Captions
\usepackage[figurename=Рисунок,labelsep=endash,
            justification=centering]{caption}
\usepackage{hyphenat}

% Lists
\usepackage{enumitem}
\renewcommand{\labelitemi}{$-$}
\setlist{nosep, leftmargin=\parindent}
\makeatletter
\AddEnumerateCounter{\asbuk}{\russian@alph}{щ}
\makeatother

% Text
\usepackage{amstext}
\usepackage{ulem}
\renewcommand{\ULdepth}{1.5pt}

% Formulas
\DeclareMathOperator*{\argmax}{argmax}

% Tabulars
\usepackage{threeparttable}
\usepackage{csvsimple}
\usepackage{longtable,ltcaption,booktabs}
\usepackage{tabularx}
\usepackage{multirow}

% PDF
\usepackage{pdfpages}

% Listings
\usepackage{listings}
\usepackage[newfloat]{minted}
\usepackage{verbatim}
\usepackage[framemethod=tikz]{mdframed}

\mdfdefinestyle{mymdstyle}{
    innerleftmargin=0mm,
    innerrightmargin=0mm,
    innertopmargin=-4pt,
    innerbottommargin=-9pt,
    splittopskip=\baselineskip,
    hidealllines=true,
    middleextra={
      \node[anchor=west] at (O|-P)
        {};
        },
    secondextra={
      \node[anchor=west] at (O|-P)
        {};},
}

\surroundwithmdframed[style=mymdstyle]{lstlisting}
\newmdenv[style=mymdstyle]{mdlisting}

\newcommand{\mylisting}[4] {
    \noindent
    {
    \captionsetup{justification=raggedright,singlelinecheck=off}
    \begin{lstinputlisting}[
        caption={#3},
        label={lst:#4},
        linerange={#2}
    ]{../../src/#1}
    \end{lstinputlisting}
}}

\newcommand{\mybreaklisting}[4]{
    \begin{mdlisting}
        \captionsetup{justification=raggedright,singlelinecheck=off}
        \lstinputlisting[label=lst:#4, caption=#3, linerange=#2]{../../#1}
    \end{mdlisting}
}


% russian language support by listings
% replace sign to text
\lstset{
	literate=
	{а}{{\selectfont\char224}}1
	{б}{{\selectfont\char225}}1
	{в}{{\selectfont\char226}}1
	{г}{{\selectfont\char227}}1
	{д}{{\selectfont\char228}}1
	{е}{{\selectfont\char229}}1
	{ё}{{\"e}}1
	{ж}{{\selectfont\char230}}1
	{з}{{\selectfont\char231}}1
	{и}{{\selectfont\char232}}1
	{й}{{\selectfont\char233}}1
	{к}{{\selectfont\char234}}1
	{л}{{\selectfont\char235}}1
	{м}{{\selectfont\char236}}1
	{н}{{\selectfont\char237}}1
	{о}{{\selectfont\char238}}1
	{п}{{\selectfont\char239}}1
	{р}{{\selectfont\char240}}1
	{с}{{\selectfont\char241}}1
	{т}{{\selectfont\char242}}1
	{у}{{\selectfont\char243}}1
	{ф}{{\selectfont\char244}}1
	{х}{{\selectfont\char245}}1
	{ц}{{\selectfont\char246}}1
	{ч}{{\selectfont\char247}}1
	{ш}{{\selectfont\char248}}1
	{щ}{{\selectfont\char249}}1
	{ъ}{{\selectfont\char250}}1
	{ы}{{\selectfont\char251}}1
	{ь}{{\selectfont\char252}}1
	{э}{{\selectfont\char253}}1
	{ю}{{\selectfont\char254}}1
	{я}{{\selectfont\char255}}1
	{А}{{\selectfont\char192}}1
	{Б}{{\selectfont\char193}}1
	{В}{{\selectfont\char194}}1
	{Г}{{\selectfont\char195}}1
	{Д}{{\selectfont\char196}}1
	{Е}{{\selectfont\char197}}1
	{Ё}{{\"E}}1
	{Ж}{{\selectfont\char198}}1
	{З}{{\selectfont\char199}}1
	{И}{{\selectfont\char200}}1
	{Й}{{\selectfont\char201}}1
	{К}{{\selectfont\char202}}1
	{Л}{{\selectfont\char203}}1
	{М}{{\selectfont\char204}}1
	{Н}{{\selectfont\char205}}1
	{О}{{\selectfont\char206}}1
	{П}{{\selectfont\char207}}1
	{Р}{{\selectfont\char208}}1
	{С}{{\selectfont\char209}}1
	{Т}{{\selectfont\char210}}1
	{У}{{\selectfont\char211}}1
	{Ф}{{\selectfont\char212}}1
	{Х}{{\selectfont\char213}}1
	{Ц}{{\selectfont\char214}}1
	{Ч}{{\selectfont\char215}}1
	{Ш}{{\selectfont\char216}}1
	{Щ}{{\selectfont\char217}}1
	{Ъ}{{\selectfont\char218}}1
	{Ы}{{\selectfont\char219}}1
	{Ь}{{\selectfont\char220}}1
	{Э}{{\selectfont\char221}}1
	{Ю}{{\selectfont\char222}}1
	{Я}{{\selectfont\char223}}1
}
% set asm listing style
\lstset{
	language={[x86masm]Assembler},
	backgroundcolor=\color{white},
	basicstyle=\footnotesize\ttfamily,
	keywordstyle=\color{blue},
	stringstyle=\color{red},
	commentstyle=\color{gray},
	numbers=left,
	numberstyle=\footnotesize,
	stepnumber=1,
	numbersep=5pt,
	frame=single,
	tabsize=4,
	captionpos=t,
	breaklines=true
}


\begin{document}

\chapter{Вопросы по темам}

\begin{enumerate}

\section{Операционные системы.\\Классификация ядер.}
    \item (2.1) Классификация операционных систем. Особенности ОС определенных
        типов.  Виртуальная машина и иерархическая машина -- декомпозиция
        системы на уровни иерархии, иерархическая структура Unix BSD,
        архитектуры ядер ОС -- определение, примеры.
    \item (10.1) Классификация ядер операционных систем. Особености ОС
        с микроядром. Три состояния блокировки процесса при передаче
        сообщений. Достоинства и недостатки микро-ядерной архитектуры,
        операционная  система Match: основные абстракции.

\section{Режимы работы. Защищенный режим. Системные таблицы.}
    \item (1.2) Три режима работы компьютера на базе процессоров Intel(x86).
        Адресация аппаратных прерываний в защищенном режиме: таблица
        дескрипторов прерываний (IDT) -- формат дескриптора прерывания,
        типы шлюзов. Пример заполнения IDT из лабораторной работы.
    \item (2.2 + 2009) Три режима работы вычислительной системы с
        архитектурой x86: особенности. Реальный режим: линия А20 --
        адресное заворачиывание. Перевод компьютера в защищенный режим.
        Линия A20 в защищенном режиме: включение и выключение линии А20
        (код из лабораторной работы). XMS.
    \item (3.2) Защищенный режим: назначение системных таблиц -- глобальной
        таблицы дескрипторов (GDT), таблицы дескрипторов прерываний (IDT),
        теневых регистров (структуры, описывающие дескрипторы GDT и IDT
        и заполнение дескрипторов в лабораторной работе по защищенному
        режиму).
    \item (18.2) Защищенный режим: перевод компьютера в защищенный режим --
        последовательность действий; реализация -- пример кода из лабораторной
        работы.

\section{Прерывания}
    \item (3.1) Прерывания: классификация. Последовательность действий при
        выполнении запроса ввода-вывода. Обработчики аппаратных прерываний:
        виды и особенности. Функции обработчика прерывания от
        системного таймера (в ОС семейства Windows и семейства Linux).
    \item (8.2) Аппаратные прерывания: задачи обработчика прерывания от
        системного таймера в защищенном режиме.
    \item (10.2) Обработчик прерывания ДОС int 8h: функции; контроллер
        прерываний -- схема, маскируемые и немаскируемые прерывания; запрет и
        разрешение маскируемых прерываний в обработчике int 8h, префиксная
        команда lock. пример кода обработчика прерывания от системного таймера
        из лабораторной работы по защищенному режиму.
    \item (14.2) Аппаратные прерывания: типы аппаратных прерываний;
        особенности. Прерываний от устройств ввода-вывода: назначение и
        аппаратная реализация. Прерывание от системного таймера в защищенном
        режиме. Пример кода обработчика прерывания от системного таймера
        из лабораторной работы по защищенному режиму.
    \item (17.2 + 2020) Прерывание от системного таймера в защищенном режиме:
        функции (по материалам лабораторной работы). Адресация прерываний от
        системного таймера в защищенном режиме (схема).
    \item (20.2) Пересчет динамических приоритетов в ОС UNIX и Windows
        (лабораторная работа).

\section{Процессы}
\subsection{Общее}
    \item (1.1) Определение ОС. Ресурсы вычислительной системы. Режимы
        ядра и задачи: переключение в режим ядра -- классификация событий.
        Процесс, как единица декомпозиции системы, диаграмма состояний
        процесса с демонстрацией действий, выполняемых в режиме ядра.
        Переключение контекса. Потоки: типы потоков, особенности каждого типа
        потоков.
    \item (6.1) Понятие процесса. Процесс как единица декомпозиции системы.
        Диаграмма состояний процесса с демонстрацией действий,
        выполняемых в режиме ядра. Планирование и диспетчеризация.
        Классификация алгоритмов Планирования. Примеры алгоритмов
        планирования, соотнесенные с типами ОС. Процессы и потоки. Типы
        потоков.
    \item (12.1) ОС с монолитным ядром. Переключение в режим ядра.
        Диаграмма состояний процесса и переход из одного состояния в другое --
        причины каждого перехода. Диаграмма состояний процесса в UNIX.
        Переключение контекста. Система прерываний.
\subsection{UNIX. Системные вызовы}
    \item (9.2) UNIX: концепция процессов: иерархия процессов, процессы -- "сироты",
        процессы "зомби", демоны; примеры из лабораторной работы (5 программ).
    \item (16.1) Процессы в UNIX: системные вызовы fork(), exec(), wait(),
        signal() -- примеры из лабораторных работ.
    \item (27.2) Лабораторная работа по UNIX: системные вызовы fork(), wait(),
        exec(), pipe(), signal(). Примеры программ из лабораторной работы.
    \item (19.1) Процессы Unix: создание процесса в ОС Unix и запуск новой
        программы. Примеры программ из лабораторных работ,
        деменстрирующие эти действия. Системные вызовы wait() и pipe():
        назначение, примеры из лабораторных работ. Процессы "сироты", "зомби"
        и "демоны".
    \item (21.2) Процессы Unix: создание процесса в ОС Unix и запуск новой
        программы. Примеры из лабораторной работы (коды).

\subsection{Взаимодействие параллельных процессов}
    \item (11.1) Параллельные процессы: взаимодействие, обоснование
        необходимости монопольного доступа к разделяемым переменным, способы
        взаимоисключения. Мониторы: определение; примеры -- простой
        монитор и монитор кольцевой буфер.
    \item (20.1) Процессы: взаимодейсвие параллельных процессов --
        монопольный доступ и взаимоисключение; программная
        реализация взаимоисключения -- примеры, семафоры -- пределение,
        виды семафоров, примеры использования множественных семафоров из
        лаборатолабораторных работ "производство-потребление" и
        "читатели-писатели"
    \item (22.1 + 2020) Процессы: взаимодействие параллельных процессов --
        монопольный доступ и взаимоисключение; аппаратная
        реализация взаимоисключения, спин-блокировка -- реализация. Семафоры
        Дейкстры: определение, взаимоисключение с помощью
        семафоров, алгоритм "Производство-потребление"\ -- решение Дейкстры.

\subsection{Параллельные процессы. Средства UNIX}
    \item (11.2) Средства межпроцессного взаимодействия оIPC) операционной
        системы UNIX System V: очереди сообщений и программные каналы --
        сравнение, примеры (для программных каналов пример из лабораторной
        работы с сигналами)
    \item (15.1) Межпроцессное взаимодейсвие в Unix System V (IPC): сигналы,
        программные каналы, семафоры и разделяемая память; примеры
        использования из лабораторных работ.

        
\subsection{Производство-потребление. Читатели-писатели.}
    \item (5.2) Задача "Производство-потребление": алгоритм Эд. Дейкстры,
        реализация на семафорах UNIX (код из лабораторной работы).
    \item (8.1) Взаимоисключение и синхронизация процессов и потоков.
        Семафоры: определение, виды. Семафор, как средство синхронизации и
        передачи сообщений. Семафоры UNIX: примеры решения задач с помощью
        семафоров: "Производство-потребление" и "Читатели-писатели" в UNIX
        (пример реализации в лабораторной работе).
    \item (6.2) Обеспечение монопольного доступа к разделяемым данным в
        задаче "писатели-читатели": реализация на базе Win32 API
        (пример кодов лабораторной работы "читатели-писатели" для ОС
        Windows). (Сравнение мьютексов и семафоров).
    \item (12.2) Задача: читатели-писатели -- монитор Хоара, решение с
        использованием семафоров Unix и разделяемой памяти, пример
        реализации из лабораторной работы.
    \item (19.2) Взаимодействие параллельных процессов: мониторы --
        определение; монотор Хоара "читатели-писатели", реализация в ОС Windows
        -- пример из лабораторной работы.
    \item (21.1 + 2020) Взаимодействие параллельных процессов: монопольное
        использование -- реализация; типы реализации взаимодействия.
        Мониторы -- определение, примеры: простой монитор, монитор
        "кольцевой буфер" и монитор "читатели-писатели". Пример реализации
        монитора "читатели-писатели" для ОС Windows. Алгорит Э. Дейкстры
        "Алгоритм банкира" и алгоритм Хабермана с примеров определения
        состояния системы.

\subsection{Проблемы распараллеливания. Тупики. Философы. Булочная. Лампорт.}
    \item (4.1) Тупики: Обнаружение тупиков для повторно используемых ресурсов
        методом редукуции графа, способы представления графа, алгоритмы
        обнаружения тупиков. Пример анализа состояния системы метод редукции
        графа. Методы восстановления работоспособности системы.
    \item (15.2 2020) Тупики: классификация ресурсов и их особености. Четыре
        условия возникновения тупика. Методы исключения тупиков.
    \item (4.2) Задача "Обедающие философы"\ -- модели распределения
        ресурсов вычислительной системы. Множественные семафоры UNIX:
        системные вызовы, поддержка в системе, пример использования из
        лабораторной работы "производство-потребление".
    \item (18.1) Процессы: взаимодействие параллельных процессов --
        монопольный доступ и взаимоисключение; программная
        реализация взаимоисключения -- флаги, алгоритм Деккера, алгоритм
        Лампорта "Булочная".
    \item (24.1) Процессы: взаимодействие параллельных процессов --
        монопольный доступ и взаимоисключение; алгорит Лампорта
        "Булочная" и "Логические часы" Лампорта.
    \item (22.2) Процессы: бесконечное откладывание, зависание, тупиковая
        ситуация -- анализ на примере задачи об обедающих философах и примеры
        аналогичных ситуаций в ОС. Множественные семафоры в Linux:
        системные вызовы и поддержка в ОС Linux; примеры из лабораторных
        работ.
    \item (24.2) Процессы: бесконечное откладывание, зависание, тупиковая
        ситуация -- анализ на примере задачи об обедающих философах и примеры
        аналогичныхх ситуаций в ОС. Множественные семафоры в Linux: системные
        вызовы и поддержка в системе; пример из лабораторной работы
        "производство-потребление".

\subsection{Параллельные процессы в рапределенных системах}
    \item (13.2) Синхронизация и взаимоисключение параллельных
        процессов в распределенных системах: централизованный и распределенный
        алгоритмы, алгоритм Token-ring; сравнение алгоритмов. Транзакции:
        определение, особенности, двухфазный протокол фиксации.
    \item (14.1) Процессы: взаимодействие процессов в распределенных
        системах; цетрализованный и распределенный алгоритмы,
        синхронизация логических часов (алгоритм Лампорта); RPC --
        мехамизм.
    \item (15.2) Синхронизация и взаимоисключение параллельных
        процессов в распределенных системах: централизованный и распределенные
        алгоритмы: сравнение.
    \item (16.1) Взаимодействие параллельных процессов: проблемы; монопольный
        доступ и взимоисключение (определения); взаимодействие параллельных
        процессов в распределенных системах -- особенности;
        централизованный алгоритм, распределенный алгоритм;
        синхронизация логических часов (алгоритм Лампорта).

\subsection{Виртуальная память}
    \item (5.1) Виртуальная память: рапределение памяти страницами по
        запросам, схема с гиперстраницами, обоснование использования данной
        схемы. Управление памятью страницами по запросам в архитетуре x86 --
        расширенное преобразование (PAE) -- схема преобразования. Анализ
        страничного поведедения процессов: свойство локальности, рабочее
        множество.
    \item (9.1 + 2009) Виртуальная память: управление памятью страницами по
        запросу -- три схемы преобразования; реализация страничного
        преобразования в компьютерах на базе процессоров Intel (x86):
        стандартное преобразование и PAE в защищенном режиме -- схемы,
        размеры таблиц и их количество на каждом этапе преобразования.
        Сегментно-страничное распределение памяти по запросам (сегментами,
        разделенными на страницы по запросу???)
    \item (13.1) Вирутуальная память: управление памятью страницами по
        запросам -- три схемы. Алгоритмы вытеснения страниц: демонстрация
        особенностей на модели траектории страниц. Рабочее множество --
        определение, глобальное и локальное размещение. Флаги в дескрипторах
        страниц, предназначенные для реализации замещения страниц.
    \item (17.1) Виртуальная память: распределение памяти страницами по
        запросам, свойство локальности, рабочее множество, анализ
        страничного поведения процессов. Схема страничного преобразования в
        процессорах Intel (x86) PAE -- размеры таблиц и дескрипторов.
        Обоснование использования многоуровневого преобразования,
        кэш TLB -- струкутра (процессор 486).
    \item (27.1) Вирутуальная память: управление памью страницами по
        запросам -- три схемы преобразования виртуального адреса к физическому.
        Алгоритмы вытеснения страниц: демонстрация особенностей на
        модели траектории траниц. Рабочее множество -- определение,
        глобальное и локальное замещение. Флаги в дескрипторах страниц,
        предназанченные для реализации замещения страниц.
    \item (7.1) Управление виртуальной памятью. Распределение памяти
        сегментами по запросам: схема преобразования виртуального
        адреса, способы организации таблиц сегментов, стратегии выбора
        разделов памяти для загрузки сегментов, алгоритмы и особенности
        замещения сегментов.
    \item (7.2) Управление памятью сегментами по запросам в архитектуре x86.
        Организация таблиц сегментов в защищенном режиме. Формат
        дескриптора сегмента в таблицах дескрипторов сегментов (GDT и LDT)
        (заполнение полей дескрипторов GDT из лабораторной работы по защищенному
        режиму).
\end{enumerate}

\end{document}
