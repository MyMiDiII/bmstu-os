\chapter{Используемые структуры}

Версия ядра: \texttt{5.15.38}

\mylisting{filename.h}

\mylisting{openflags.h}

\mylisting{auditnames.h}

\clearpage
\mylisting{nameidata.h}

\mylisting{openhow.h}

\mylisting{path.h}

\clearpage
\section*{Флаги системного вызова \texttt{open()}}

\texttt{O\_CREAT} --- если файл не существует, то он будет создан.

\texttt{O\_EXCL} --- если используется совместно с \texttt{O\_CREAT}, то при
наличии уже созданного файла вызов завершится ошибкой.

\texttt{O\_NOCTTY} --- если файл указывает на терминальное устройство, то оно не
станет терминалом управления процесса, даже при его отсутствии.

\texttt{O\_TRUNC} --- если файл уже существует, он является обычным файлом и
заданный режим позволяет записывать в этот файл, то его длина будет урезана до
нуля.

\texttt{O\_APPEND} --- файл открывается в режиме добавления, перед каждой
операцией записи файловый указатель будет устанавливаться в конец файла.

\texttt{O\_NONBLOCK}, \texttt{O\_NDELAY} --- файл открывается, по возможности, в
режиме non-blocking, то есть никакие последующие операции над дескриптором файла
не заставляют в дальнейшем вызывающий процесс ждать.

\texttt{O\_SYNC} ---  файл открывается в режиме синхронного ввода-вывода, то
есть все операции записи для соответствующего дескриптора файла блокируют
вызывающий процесс до тех пор, пока данные не будут физически записаны

\texttt{O\_NOFOLLOW} --- если файл является символической ссылкой, то open
вернёт ошибку.

\texttt{O\_DIRECTORY} --- если файл не является каталогом, то open вернёт
ошибку.

\texttt{O\_LARGEFILE} --- позволяет открывать файлы, размер которых не может
быть представлен типом off\_t (long).

\texttt{O\_DSYNC} ---  операции записи в файл будут завершены в соответствии с
требованиями целостности данных синхронизированного завершения ввода-вывода.

\texttt{O\_NOATIME} ---  запрет на обновление времени последнего доступа к файлу
при его чтении.

\texttt{O\_TMPFILE} --- при наличии данного флага создаётся неименованный
временный обычный файл.

\texttt{O\_CLOEXEC} --- включает флаг close-on-exec для нового файлового
дескриптора, указание этого флага позволяет программе избегать дополнительных
операций \texttt{fcntl F\_SETFD} для установки флага \texttt{FD\_CLOEXEC}.
