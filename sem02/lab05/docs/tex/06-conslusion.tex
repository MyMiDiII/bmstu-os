\chapter{Вывод}

В данной лабораторной работе демонстрируется ряд проблем при работе с
вводом-выводом.

В первой программе предполагалось, что символы из файла будут читаться по
очереди каждым из потоков и выведутся в том порядке, в котором они
расположены в файле, однако включение буферизации, как было показано,
привело к другому результату (проблема буфферизации).

Во второй программе в силу наличия двух файловых дескрипторов, связанных с одним
\texttt{inode}, чтение файла и вывод информации из него происходит дважды. При
необходимости решения этой проблемы можно создать разделяемую область памяти для
отслеживания позиции в файле и ипользовать \texttt{mutex} для доступа к ней.

В третьей программе, так же как и в первой, показывается проблема буферизации,
из-за которой в данной программе запись в файл происходит только при вызове
\texttt{fclose()}, данные в файле перезаписываются, возникает потеря информации.
Решением проблемы является открытие файла в режиме добавления
\texttt{O\_APPEND}.  В этом случае операция записи в файл атомарна, а перед
каждым вызовом функции записи смещение в файле будет устанавливаться на его
конец.
